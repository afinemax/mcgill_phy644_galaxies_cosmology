\section*{Problem 2: The Plummer Potential}

The Plummer gravitational potential is:
\begin{equation}
    \Phi(r) = \frac{-GM}{({r^2 + r_0^2})^{1/2}}
\end{equation}
where \(r\) is the distance from the center, \(M\) is the total mass of the galaxy cluster, \(r_0\) is the characteristic radius, and \(G\) is Newton's gravitational constant.


\subsection*{Problem 2A}
We are asked to derive $\rho(r)$ --- the mass density of the Plummer potential.  My idea here is to use Gauss’s law for gravity in differential form! It looks like this (It's in Griffith's EM):

\begin{equation} \label{eq:gauss_law}
\nabla \cdot \mathbf{g} = -4 \pi G \rho
\end{equation}

Recall that $ \mathbf{g}(r) = - \nabla \Phi(r) $, putting this into Equation \ref{eq:gauss_law} we have:

\begin{equation}\label{eq:laplace_with_G}
\nabla \cdot (- \nabla \Phi) = -4 \pi G \rho \quad \Rightarrow \quad \nabla^2 \Phi = 4 \pi G \rho
\end{equation}

Recall for a spherically symmetric potential, the Laplacian in spherical coordinates is:
\begin{equation} \label{eq:laplace_with_phi}
\nabla^2 \Phi = \frac{1}{r^2} \frac{d}{dr} \left( r^2 \frac{d\Phi}{dr} \right)
\end{equation}

Now we just need to take a few derivatives and rearrange! I wonder if there is a faster way. Anyway lets start with $\frac{d\Phi}{dr}$:

\begin{equation}
\frac{d\Phi}{dr} = \frac{GM r}{(r^2 + r_0^2)^{3/2}}
\end{equation}

The term in () in Equation \ref{eq:laplace_with_phi} is then:

\begin{equation}
\Rightarrow \frac{GM r^3}{(r^2 + r_0^2)^{3/2}}
\end{equation}

%%%%%%%%%%%%%%%%%%% New here

Taking the next derivative we get:
\begin{equation}
\Rightarrow GM\frac{3 r^2 r_0^2}{(r^2 + r_0^2)^{5/2}}
\end{equation}

Tossing in the factor of $1/r^2$, the Laplacian aka Equation \ref{eq:laplace_with_phi} is:
\begin{equation}
\nabla^2 \Phi = \frac{3 GM r_0^2}{(r^2 + r_0^2)^{5/2}}
\end{equation}


Finally, using Equation \ref{eq:laplace_with_G}, we solve for $\rho(r)$ -  the mass density profile of the Plummer potential.
\begin{equation}
\boxed{
\rho(r) = \frac{3 M r_0^2}{4 \pi (r^2 + r_0^2)^{5/2}}
}
\end{equation}

Factoring the $r_0$ we get the form from the problem set.
\begin{equation}
\boxed{
\rho(r) = \frac{3 M}{4 \pi r_0^3 \left( 1 + \left(\frac{r}{r_0}\right)^2 \right)^{5/2}}
}
\end{equation}


\subsection*{Problem 2B}

This problem asks us to find the surface mass density projected onto the sky. After consultation with the TA, this means that we integrate the 3D mass density along the line-of-sight to get a 2D (projected) mass density. 




\begin{equation}
\Sigma(R) = \int_{-\infty}^{\infty} \rho\!\left(R^{2} + z^{2}\right) \, dz
\end{equation}
here $z$ is not redshift, but the line-of-sight distance.

We can relate $r$, $R$, and $z$ as $r^2 = R^2 + z^2$. Big $R$ is the projected is the projected radius from the centre of the cluster.

Expressing $\rho\!\left(R^{2} + z^{2}\right)$:
\begin{equation}
\rho = \frac{3 M }{4\pi r_0^3} (1 +  \frac{R^2 + z^2}{r_0^2})^{-5/2}
\end{equation}

Substituting into the integral:

\begin{equation}
\Sigma(R) = \frac{3 M }{2\pi r_0^3} 
\int_{0}^{\infty} \left( 1 + \frac{R^2 + z^2}{r_0^2} \right)^{-\tfrac{5}{2}} \, dz
\end{equation}

Using trusty wolfram alpha as our integral table:

\begin{equation}
\Sigma(R) = \frac{3 M }{2\pi r_0^3} \frac{2r_0^5}{3(r_0^2 + R^2)^2}
\end{equation}

Simplifying: 

\begin{equation}
\boxed{\Sigma(R) = \frac{M}{\pi r_0^2}\frac{1}{\left[1 + \left(\tfrac{R}{r_0}\right)^2\right]^2}}
\end{equation}

\subsection*{Problem 2C}
This is really 3 problems, but they are fairly simple. We are told to assume the globular cluster is made of identical stars and that the cluster contains no dark matter, and use units of $r_0$. 


\subsubsection*{2C I}

We are asked to compute the Core radius $r_c$ defined as where the surface density falls to half its central value. This is simply the condition of $\Sigma(r_c) = \frac{1}{2}\Sigma(0)$.

\begin{equation}
\frac{M}{\pi r_0^2}\frac{1}{\left[1 + \left(\tfrac{r_c}{r_0}\right)^2\right]^2} =  \frac{1}{2} \frac{M}{\pi r_0^2}\frac{1}{\left[1 + \left(\tfrac{0}{r_0}\right)^2\right]^2}
\end{equation}

Lots of things cancel!

\begin{equation}
\Rightarrow \frac{1}{\left[1 + \left(\tfrac{r_c}{r_0}\right)^2\right]^2} = \frac{1}{2}
\end{equation}

\begin{equation}
\Rightarrow \left[1 + \left(\tfrac{r_c}{r_0}\right)^2\right]^2 = 2
\end{equation}

\begin{equation}
\Rightarrow 1 + \left(\tfrac{r_c^2}{r_0^2}\right) = \sqrt{2}
\end{equation}


\begin{equation}
\boxed{r_c = r_0((\sqrt{2} -1 ))^{0.5}} \approx 0.64r_0
\end{equation}


\subsubsection*{Problem 2C II}
Now we calculate the Half-light radius, within which half of the total light is projected. I'm going to call this $H_{0.5}$ (H for half?). (we are assuming 0 extinction as well).

Assuming globular cluster is made of identical stars and that the cluster contains no dark matter, and use units of $r_0$.  We can relate the $\Sigma(R)$ to surface brightness density $B(R)$. 

The total mass $M$ is given by $ M = Nm$, where $N$ is the number of stars and $m$ is the mass of the stars (in our case a constant). We also know that mass-to-luminosity ratio is a constant either $k = \frac{m}{l}$ per stare or $K = \frac{M}{L}$ for the cluster.

Multiplying $\Sigma$ by $k$ should give us the luminosity density. Aka we just replace $M$ with $KL$ in $\Sigma $ or $\sigma$. Then we integrate to $H_{0.5}$. 

\begin{equation}
B(R) = \frac{KL}{\pi r_0^2}\frac{1}{\left[1 + \left(\tfrac{R}{r_0}\right)^2\right]^2}
\end{equation}

Now we integrate $B(R)$ $0$ to $2$ $\pi$, and over R. $da = RdRd\theta$ - the surface of our projected circle blob. 
\begin{equation}
\frac{K}{2} = \frac{2KL}{ r_0^2} \int_{0}^{H_{0.5}} R\frac{1}{\left[1 + \left(\tfrac{R}{r_0}\right)^2\right]^2} \, dR
\end{equation}

Using wolfram alpha:
\begin{equation}
\frac{K}{2} =   \frac{2K}{r_0^2} \frac{H_{0.5}^2r_0^2}{2(r_0^2 + H_{0.5}^2)}
\end{equation}

Now lets go cancel a lot of things
\begin{equation}
    \Rightarrow \frac{1}{2} = \frac{H_{0.5}^2}{(r_0^2 + H_{0.5}^2)}
\end{equation}

\begin{equation}
    \Rightarrow r_0^2 + H_{0.5}^2 = 2H_{0.5}^2
\end{equation}

\begin{equation}
    \Rightarrow r_0^2 = 2H_{0.5}^2 -  H_{0.5}^2   = H_{0.5}^2
\end{equation}


\begin{equation}
    \boxed{H_{0.5} = r_0}
\end{equation}
The Half-Right radius is the $r_0$ characteristic radius!

\subsubsection*{Problem 2C III}

Half-mass radius, the radius of the 3D sphere which contains half the cluster’s total mass. This is very similar to the previous part, we use the same integral method but on $\rho(r)$, and relate it to $\frac{M}{2}$. Let's use $r_h$ for the half-mass radius:

\begin{equation}
\frac{M}{2} = \int_0^{r_h} 4\pi r^2 \rho(r) \, dr ,
\end{equation}
the $4\pi$ comes from being spherically symmetrical. 

Plugging in our $\rho(r)$:
\begin{equation}
\frac{M}{2} = \int_0^{r_h} 4\pi r^2 \frac{3 M r_0^2}{4 \pi (r^2 + r_0^2)^{5/2}} \, dr ,
\end{equation}

Moving stuff to make it easier to use wolfram alpha:
\begin{equation}
\frac{1}{2} = 3 \int_0^{r_h}  r^2 \frac{ r_0^2}{ (r^2 + r_0^2)^{5/2}} \, dr ,
\end{equation}

Now using wolfram alpha:
\begin{equation}
\frac{1}{2} = 1 - \frac{1}{\left(1 + \frac{r_h^2}{r_0^2}\right)^{3/2}}
\end{equation}

Simplifying: 
\begin{equation}
\boxed{r_h = \frac{r_0}{\sqrt{2^{2/3} -1}} \approx 0.76 r_0} 
\end{equation}



