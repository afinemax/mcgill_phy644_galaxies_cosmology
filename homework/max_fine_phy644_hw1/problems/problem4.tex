\section*{Problem 4: The Tully-Fisher Relation}

the Tully-Fisher relation, which states that the
luminosity L of a spiral galaxy is roughly proportional to the circular velocity $v_c$ of stars to the fourth power:

\begin{equation}
    L \propto v^4_c.
\end{equation}

This used to be used to infer distances on the cosmic distance ladder until better techniques arised, the Tully-Fisher relation has a $\sim 10\%$ scatter. 

\subsection*{Problem 4A}

The Virial Theorem states $2K + V = 0$, where $K$, and $V$ are the kinetic and potential energy respectfully. 

In our case we can write:
\begin{equation}
    Mv_c^2 + \frac{\alpha GM^2}{R} = 0
\end{equation}
here, M is the mass the galaxy, and R is the distance from the centre, $\alpha$ is some correction factor.

We can rewrite this as:
\begin{equation}
    v_c^2 \propto \frac{M}{R}
\end{equation}


For the case of a constant mass-luminosity ratio ($M/L = c_1$), we can replace $M$ in terms of $L$. 

\begin{equation} \label{problem_2_a}
    \boxed{v_c^2 \propto \frac{L}{R}}
\end{equation}

\subsection*{Problem 4B}
Here we now also assume constant surface brightness $c_2 = \frac{L}{A}$, where A is area. We can write $A = \alpha R^2$, where again $\alpha$ is some correction factor. Using our solution to problem 2A equation \ref{problem_2_a}, we can write:

\begin{equation}
    R \propto L^{1/2}
\end{equation}

and subsisting into equation \ref{problem_2_a},

\begin{equation}
    v_c^2 \propto L^{1/2}
\end{equation}

rewriting:
\begin{equation}
    \boxed{L \propto v_c^4}
\end{equation}


\section*{Problem 4C}
Assumptions made in this problem were 1) Virial Theorem holds, 2) constant Mass-to-Luminosity ratio, 3) constant surface brightness, 4) no darkmatter.


1) Virtualisation holds for bound systems like some atoms, globular cluster, (closed) planetary orbits. For galaxies - it doesn't hold as they have feedback and evolution. 

2) is unrealistic unless there is no active star-formation. We know that for main sequence stars there $L \sim M^\alpha$, with $\alpha = 3.5$ - a power-law not a constant. 

3) unrealistic, galaxies are brighter at their center and dimmer at their edges

4) We never invoked the existence of Dark matter into our analysis, at least to my understanding the virtualisation I used is for normal matter.  

