

\section*{Problem 1: Free-fall Time}


We are asked to derive the true free fall time $t_{ff}$ of presser-less dust ball of uniform density $\rho$ collapsing.

The total mass of the sphere is :
\begin{equation} \label{eq:mass_enclosed}
    M = \frac{4}{3}\pi \rho r_0^3
\end{equation}

where $M$ is the total mass, and $r_0$ is the initial radius (max radius). The  $t_{ff}$ is the time it takes for a test mass on the surface to fall to the centre. 

We know from Guass's law for gravity that this problem is the equivalent of asking how long it takes for a test mass to fall into the body it is orbiting. 

From Kepler's 3rd law we know that:

\begin{equation} \label{keplers 3rd law}
    \frac{P^2}{a^3} = \frac{4\pi^2}{G(M + u)}
\end{equation}

Where $P$ is the period of the orbit, $a$ is the semi-major axis, $M$ mass of the large body and $u$ is our test mass. $M >> u$, so we can neglect $u$. This is true regardless of the eccentricity ($e$) of the orbit, for the special case of $e=1$ (meaning $a = \frac{1}{2}r_0$), the orbit is a line. The $t_{ff}$ is then interpreted as $\frac{1}{2}P$. Substituting into equation  \ref{keplers 3rd law}:

\begin{equation} 
    \frac{t_{ff}^2}{r_0^3} = \frac{\pi^2}{8GM}
\end{equation}

\begin{equation} 
    t_{ff}^2 = \frac{\pi^2r_0^3}{8GM}
\end{equation}


\begin{equation} 
    t_{ff} = (\frac{\pi^2r_0^3}{8GM})^{\frac{1}{2}}
\end{equation}

Now we replace $M$ with equation \ref{eq:mass_enclosed}. 
\begin{equation} 
    t_{ff} = (\frac{\pi^2r_0^3}{8G\frac{4}{3}\pi \rho r_0^3})^{\frac{1}{2}}
\end{equation}

Lots of things cancel!

\begin{equation} 
    t_{ff} = (\frac{3\pi}{32G\ \rho })^{\frac{1}{2}}
\end{equation}

We can factor out a $\frac{1}{\sqrt{16}}$

\begin{equation} 
    \boxed{t_{ff} = \frac{1}{4}\sqrt{\frac{3\pi}{2G\ \rho }}}
\end{equation}






The hard way is using conservation of energy, which I will also do. I am assuming that energy conservation holds, for a test mass at the edge of the surface
\begin{equation}
    E = \frac{1}{2}v_0^2 - \frac{GM}{r_0} = \frac{1}{2}v(r)^2 - \frac{GM}{r} 
\end{equation}
where $E$ is a constant, and this is the per unit mass energy. We take $v_0$ to be 0. 

We can rearrange for $v(r)$:

\begin{equation}
    v(r)^2 = 2GM(\frac{1}{r} - \frac{1}{r_0}) 
\end{equation}

now we have a first order differential equation: 
\begin{equation}
    \frac{dr}{dt} = -[2GM(\frac{1}{r} - \frac{1}{r_0})]^{0.5}
\end{equation}
with the same initial conditions, the $-$ comes from falling inwards. 

\begin{equation}
     -[2GM(\frac{1}{r} - \frac{1}{r_0})]^{-0.5} dr = dt
\end{equation}

The integral bounds are from $r_0$ to $0$ on the left hand side and from $ 0$ to $t_{ff}$ on the right hand side

\begin{equation}
    \int_{r_0}^{0} 
    -\left[2GM\!\left(\frac{1}{r}-\frac{1}{r_0}\right)\right]^{-1/2} dr 
    = \int_{0}^{t_{ff}} dt
\end{equation}

\begin{equation}
    \int_{0}^{r_0} 
    \left[2GM\!\left(\frac{1}{r}-\frac{1}{r_0}\right)\right]^{-1/2} dr 
    = t_{ff}
\end{equation}

\sout{Now we use our integral table aka wolfram alpha (it looks like a $u$ and then trig sub).}

Before we can use an integral table, we need to simplify more, let $u = \frac{r}{r_0}$, $du = \frac{1}{r_0}dr$. 

\begin{equation}
    \int_{0}^{1} 
    [2GM(\frac{1-u}{ur_0})]^{-1/2} \frac{1}{r_0}du 
    = t_{ff}
\end{equation}







