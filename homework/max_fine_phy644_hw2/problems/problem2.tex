\section*{Problem 2: Virial Theorem}

We are asked to show that the Virial theorem is:

\begin{equation}
    U + 2T = \frac{1}{2}\frac{d^2I}{dt^2},
\end{equation}
where U is the potential energy, T is the kinetic energy, and I is the moment of inertia $I = \sum_i m_ir_i^2$, where $r$ is the position, and $m$ is the mass of each particle. 

We are given a hint to use $G$:

\begin{equation}
    G = \sum_i\boldsymbol{P}_i \cdot \boldsymbol{r}_i,
\end{equation}
where $\boldsymbol{P}$ is the momentum of each particle. 


$G$, and $I$ are related:
\begin{equation}
    G = \sum_i\boldsymbol{P}_i \cdot \boldsymbol{r}_i = \sum_im_i \boldsymbol{v}_i \cdot r_i = \frac{1}{2}\sum_im_i \frac{d}{dt}(r_i^2) = \frac{1}{2}\frac{dI}{dt}
\end{equation}

Now lets take a second $\frac{d}{dt}$ of $G$, we need to find the Left hand side of the following equation and we are done. 
\begin{equation}
    \frac{dG}{dt} = \frac{d}{dt}(\frac{1}{2}\frac{dI}{dt}) 
\end{equation}

Let's start:

\begin{equation}
    \frac{dG}{dt} = \frac{d}{dt}\sum_im_i \boldsymbol{v}_i \cdot r_i = \sum_i F_i \cdot r_i + \sum_i m_iv_i^2 = \sum_i F_i \cdot r_i  + 2T
\end{equation}

$\sum_i F_i \cdot r_i = U$ is the definition of potential energy!

therefor:
\begin{equation}
    \boxed{U + 2T =  \frac{1}{2}\frac{d^2I}{dt^2} }
\end{equation}









