\section*{Problem 3: Singular Isothermal Spheres}

Suppose that the probability distribution of velocities of particles (e.g., stars) in a galaxy are given by a
Maxwell-Boltzmann distribution, where the velocities are Gaussian:


\begin{equation} \label{eq:maxwell-boltzmann_dist}
p(\mathbf{v}) = \frac{1}{(2 \pi \sigma^2)^{3/2}} 
\exp\left( -\frac{|\mathbf{v}|^2}{2 \sigma^2} \right),
\end{equation}
Here, $\mathbf{v}$ represents the 3D velocity vector 
$(v_x, v_y, v_z)$, and $|\mathbf{v}|$ is its magnitude $v$.


\subsection*{Problem 3A:}
We are asked to show that the standard deviation of Equation \ref{eq:maxwell-boltzmann_dist} is $\sigma$. 

The symmetrical 3D Gaussian factorizes:
\begin{equation}
p(\mathbf{v}) = \frac{1}{(2 \pi \sigma^2)^{3/2}} 
\exp\left( -\frac{{v_x}^2 + {v_y}^2 + {v_z}^2}{2 \sigma^2} \right),
\end{equation}

In any single dimension the Maxwell-Boltsman equation is therefor: 

\begin{equation}
p({v}) = \frac{1}{(2 \pi \sigma^2)^{3/2}} 
\exp\left( -\frac{{v}^2 }{2 \sigma^2} \right),
\end{equation}

The variance is defined as $\mathrm{Var}(v) = \mathbb{E}[v^2] - (\mathbb{E}[v])^2$. (Variance is the square of the STD). In our cause $\mathbb{E}[v] = 0$ due to symmetry. 

$\mathbb{E}[x^2]$ is given by:

\begin{equation}
    \mathbb{E}[v^2] =
\int_{-\infty}^{\infty} v^2p(v)dv %e^{-x^2} \, dx = \sqrt{\pi}
\end{equation}

Substituting in our $p(v$
\begin{equation}
    \mathbb{E}[v^2] =
\int_{-\infty}^{\infty} v^2\frac{1}{(2 \pi \sigma^2)^{3/2}} 
\exp\left( -\frac{{v}^2 }{2 \sigma^2} \right)dv %e^{-x^2} \, dx = \sqrt{\pi}
\end{equation}


Integrals of this type are known as Gaussians, and their solution is well known. 

\begin{equation}
    \int_{-\infty}^{\infty} x^2 e^{-a x^2} \, dx = \frac{\sqrt{\pi}}{2 a^{3/2}}, \quad a > 0
\end{equation}
    In our case, $a = \frac{1}{2 \sigma^2}$. 

Clearly this evaluates to $\boxed{\mathbb{E}[v^2]  = \sigma^2}$.


\subsection*{Problem 3B:}
Write down a differential equation for hydrostatic equilibrium of a spherically symmetric system in terms of the radial coordinate $r$, the density $\rho$ and pressure $P$ (both of which can be functions of $r$), $G$, and $M_r$ (the total mass enclosed internal to radius $r$).


From dimensional analysis and Newton's law of gravity, Hydrostatic equilibrium follows:
\begin{equation} \label{eq:hydrostatic}
    \boxed{\frac{dP}{dr} = -\rho(r) \frac{GM(r)}{r^2}}
\end{equation}
where $M$ is the enclosed mass. 

(we are not asked to\textit{ solve} the differential equation, only to write it)


\subsection*{Problem 3C:}
For an isothermal, isotropic velocity distribution the pressure is
\begin{equation}
    P = \rho\sigma^2 
\end{equation}

We are asked now to solve the differential equation :( . Rewriting Equation \ref{eq:hydrostatic}, and using $\frac{dM}{dr} = 4\pi r^2 \rho $, (its easier with the $\ln$ according to some online textbook I found)

\begin{equation}\label{eq:ode_3c}
    \frac{\sigma^2}{r^2} \frac{d}{dr}(r^2 \frac{d\ln{\rho}}{dr}) = -4\pi G \rho(r)
\end{equation}

Ansatz for a power-law solution:
\begin{equation}
    \rho(r) = \frac{A}{r^n},
\end{equation}
$A$ is a constant with appropriate units, and  I require $A>0$, $n>0$.

This means that:
\begin{equation}
    \frac{d\ln{\rho}}{dr} = \frac{-n}{r}
\end{equation}

The full left hand side (LHS) of Equation \ref{eq:ode_3c} is:
\begin{equation}
    \mathrm{LHS} = \frac{-\sigma n}{r^2}
\end{equation}

The equation is now:
\begin{equation}
   \frac{-\sigma^2 n}{r^2} = -4\pi G \frac{A}{r^n}
\end{equation}
If this holds, then $n=2$, and $A= \frac{\sigma^2}{2\pi G}$.

and therefor:
\begin{equation} \label{eq:3c_answer}
    \boxed{\rho(r) = \frac{\sigma^2}{2\pi G} \frac{1}{r^2}},
\end{equation}
which is singular at $r=0$. 

\subsection*{Problem 3D:}
We are asked to find the circular velocity $v_c$ of a test particle placed in a circular orbit. 


The circular orbit is given by:

\begin{equation}
    v_c^2 = \frac{GM(r)}{r}
\end{equation}
where $M(r)$ is the enclosed mass. 

$M(r)$ is given by:
\begin{equation}
    M(r) = 4 \pi \int_0^r \rho(r) \, {r}^2 \, dr.
\end{equation}


We know $\rho(r)$, its our answer to problem 3C equation \ref{eq:3c_answer}. 

So the circular orbit is given by:
\begin{equation}
    v_c^2 = \frac{G}{r} 4 \pi \int_0^r \frac{\sigma^2}{2\pi G} \frac{1}{r^2} \, {r}^2 \, dr
\end{equation}

This looks scary but its really easy, just with a lot of constants, solving we have the solution. 

\begin{equation} \label{eq:v_c}
    \boxed{v_c = \sqrt{2}\sigma}
\end{equation}
Woah! This is actually a super cool result! 


\subsection*{Problem 3E:}

A more realistic $\rho(r)$ is Navarro-Frenk-White (NFW) given by:
\begin{equation} \label{eq:NFW}
    \rho(r) = \frac{\rho_0}{(\frac{r}{r_s})(1 + \frac{r}{r_s})^2}
\end{equation}
where $\rho_0$ is an overall normalization and $r_s$ is known as the scale radius.

The question is how does the steepness of the NFW profile compare to that of the singular isothermal sphere at small
$r$? At large $r$?. This is pretty straight forward comparison of $\frac{r}{r_S}$. 

At small $r$, $\frac{r}{r_s} <<1$, and so $(1 + \frac{r}{r_s})^2 \approx 1$ NFW becomes:

\begin{equation}
     \rho(r) = \frac{\rho_0r_s}{r}
\end{equation}
Here the scaling is $\propto \frac{1}{r}$, which is \textbf{shallower} then the isothermal $\frac{1}{r^2}$. 

On the other end, at large $r$, $\frac{r}{r_s} >> 1$ so $(1 + \frac{r}{r_s})^2 \approx \frac{r^2}{r_s^2}.$  So NFW becomes: 
\begin{equation}
     \rho(r) = \frac{\rho_0r_s^3}{r^3}
\end{equation}
Here the scaling is $\propto \frac{1}{r^3}$, which is \textbf{steeper} then the isothermal $\frac{1}{r^3}$. 





