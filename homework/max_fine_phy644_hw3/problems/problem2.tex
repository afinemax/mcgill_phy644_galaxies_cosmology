
\section{Problem 2: Spherical Tophat Model}

\subsection*{Problem 2A:}

We are asked to solve this ODE for $\rho = f(t)$:
\begin{equation}
    \frac{1}{2}(\frac{dr}{dt})^2  - \frac{GM}{r} = \epsilon
\end{equation}

with $\epsilon = \frac{E}{m}$, in our universe a good model is that $\epsilon = 0$ --- it has just enough energy to escape. 


\begin{equation}
    \frac{1}{2}(\frac{dr}{dt})^2  - \frac{GM}{r} = 0
\end{equation}


\begin{equation}
   r (\frac{dr}{dt})^2=  2GM 
\end{equation}


\begin{equation}
   r^{1/2} \frac{dr}{dt} = (2GM)^{1/2} 
\end{equation}

\begin{equation}
   r^{1/2} dr = (2GM)^{1/2} dt
\end{equation}

Integrate both sides from $0$ to $r$, and $0$ to $t$

\begin{equation}
  \frac{2}{3} r ^{\frac{3}{2}} = [2GM]^{\frac{1}{2}}t
\end{equation}

\begin{equation}
  r^3 =  \frac{9}{2} GMt^2
\end{equation}

We now replace $M = \frac{4}{3} \pi \rho r^3$

\begin{equation}
  r^3 =  \frac{9}{2} G\frac{4}{3} \pi \rho r^3t^2
\end{equation}

\begin{equation}
  1 =  6 G \pi \rho t^2
\end{equation}


\begin{equation}
  \boxed{\rho =  \frac{1}{6G \pi t^2}}
\end{equation}


\section*{Problem 2B}

Negative.

 
Reason: $\epsilon = \frac{E}{m}$ is the total energy per unit mass. 
In the background borderline case we had $\epsilon = 0$ 
(just enough energy to escape to infinity). 
An overdensity that will eventually stop expanding and collapse because it has slightly more attraction, it is gravitationally bound, so its total energy is negative. 
You can see this from the energy equation: at turnaround $\dot{r} = 0$ so

\begin{equation}
    \epsilon = -\frac{GM}{r_{\rm max}} < 0,
\end{equation}

hence $\epsilon$ must be negative for a region that will turn around and collapse into a halo.



\section*{Problem 2C}

Start from the energy equation for a spherical shell and assume the shell is bound so
\(\varepsilon<0\). Write \(\varepsilon=-|\varepsilon|\). Then
\[
\frac{1}{2}\dot r^{2}-\frac{GM}{r}=\varepsilon=-|\varepsilon|
\quad\Longrightarrow\quad
\dot r^{2}=\frac{2GM}{r}-2|\varepsilon|.
\]

Define the constant
\[
A\equiv\frac{GM}{2|\varepsilon|}
\]
and introduce the parameter \(\eta\) by the ansatz
\[
r(\eta)=A(1-\cos\eta).
\]
Differentiate with respect to \(\eta\):
\[
\frac{dr}{d\eta}=A\sin\eta.
\]

Using \(GM=2|\varepsilon|A\) and \(r=A(1-\cos\eta)\) in the expression for \(\dot r^{2}\) gives
\[
\dot r^{2}=2|\varepsilon|\left(\frac{2A}{r}-1\right)
=2|\varepsilon|\frac{1+\cos\eta}{1-\cos\eta}.
\]
Hence
\[
\dot r=\sqrt{2|\varepsilon|}\,\sqrt{\frac{1+\cos\eta}{1-\cos\eta}}.
\]

Now compute \(dt/d\eta\) from \(dt/d\eta=(dr/d\eta)/\dot r\):
\[
\frac{dt}{d\eta}
=\frac{A\sin\eta}{\sqrt{2|\varepsilon|}\,\sqrt{\dfrac{1+\cos\eta}{1-\cos\eta}}}
=\frac{A}{\sqrt{2|\varepsilon|}}\; \sin\eta\sqrt{\frac{1-\cos\eta}{1+\cos\eta}}.
\]
Using the trigonometric identity
\(\sin\eta\sqrt{\dfrac{1-\cos\eta}{1+\cos\eta}}=1-\cos\eta\),
we obtain
\[
\frac{dt}{d\eta}=\frac{A(1-\cos\eta)}{\sqrt{2|\varepsilon|}}.
\]

Integrate from \(\eta=0\) (where \(t=0\)) to general \(\eta\):
\[
t(\eta)=\frac{A}{\sqrt{2|\varepsilon|}}(\eta-\sin\eta).
\]

Finally substitute back \(A=GM/(2|\varepsilon|)\) to express the solution in terms of \(GM\) and \(|\varepsilon|\):
\[
\boxed{%
r(\eta)=\frac{GM}{2|\varepsilon|}\,(1-\cos\eta),\qquad
t(\eta)=\frac{GM}{(2|\varepsilon|)^{3/2}}\,(\eta-\sin\eta)\,.}
\]

These are the standard cycloidal parametric equations for a bound shell with \(r(0)=0,\;t(0)=0\),
turnaround at \(\eta=\pi\) (where \(r_{\max}=2A=GM/|\varepsilon|\)), and recollapse at \(\eta=2\pi\).

