\section*{Problem 1: Density of a Self-Gravitating Disk}

Here we consider an infinite disk of stars of identical mass, $m_{*}$, in the xy plane. Assume the stars are in equilibrium
(their phase space is in steady state).

\subsection{Problem 1A}



The Jeans equation from class is:

\begin{equation}
    \partial_t \langle{\vec{v_j}} \rangle + \sum_i \langle{\vec{v_i}} \rangle 
    \vec{\nabla}_{x, i} \langle{\vec{v_j}} \rangle = - \vec{\nabla}_{x, j} \Phi - \sum_i \frac{\vec{\nabla}_{x, i} (n\sigma_{ij}^2)}{n}
\end{equation}
From left to right, we label the terms as - Bulk accretion, velocity sheer, grav force, and pressure. We are asked to find  $n(z)$ in terms of the velocity dispersion in the $\hat{z}$ direction $\sigma_z^2$, $\Phi$, and midplane density $n(0)$.
    
Since we are in a steady state, $\partial_t \langle{\vec{v_z}} \rangle = 0$

In $\hat{z}$ we have: 
\begin{equation}
    0 = -\frac{d\Phi}{dz} - \frac{1}{n}\frac{d(n\sigma^2_z)}{dz}
\end{equation}

This looks like a straight forward differential equation, let's attack it. 

\begin{equation}
    \frac{d(n\sigma^2_z)}{dz} = - n\frac{d\Phi}{dz}
\end{equation}


\begin{equation}
    \frac{1}{n\sigma^2_z}\frac{d(n\sigma^2_z)}{dz} = -  \frac{1}{\sigma^2_z}\frac{d\Phi}{dz}
\end{equation}

Switching to $\ln{n \sigma_z^2}$:

\begin{equation}
    \frac{d\ln{(n\sigma^2_z)}}{dz} = -  \frac{1}{\sigma^2_z}\frac{d\Phi}{dz}
\end{equation}

Now we integrate both sides from $0$ to $z$. 

\begin{equation}\label{eq:solution_1a}
    \boxed{\ln(\frac{n(z)\sigma^2_z}{n(0)\sigma^2_0}) = -\int_0^{\infty} \frac{1}{\sigma^2_z} \frac{d\Phi}{dz} dz}
\end{equation}

We can rearrange and solve for $n(z)$ but it look a bit ugly

\begin{equation}\label{eq:solution_1a_but_better}
    \boxed{n(z) = n(0) \frac{\sigma^2_z(0)}{\sigma^2_z(z)} \exp(-\int_0^{\infty} \frac{1}{\sigma^2_z} \frac{d\Phi}{dz} dz)}
\end{equation}


\subsection*{Problem 1B}
In the case of an isothermal gas, and assuming $\sigma_z^2 = C$ a constant in $z$, and setting $\Phi(0) = 0$. 

The right hand side in \ref{eq:solution_1a_but_better} simplifies:

\begin{equation}
    \boxed{n(z) = n(0) e^{-\frac{\Phi(z)}{\sigma_z^2}}}
\end{equation}


Interpreting this as a thermal equilibrium (Boltzmann) distribution for a ``gas'' of particles of mass  $m_*$, the velocity dispersion plays the role of the thermal kinetic energy per unit mass. The effective temperature $T$ is given by:

\begin{equation}
    \frac{1}{2}m_*\langle v^2\rangle \sim \frac{1}{2}m_*\sigma^2_z \sim \frac{1}{2}k_bT
\end{equation}

So the temperature of the gas is given by:

\begin{equation}
    \boxed{T = \frac{m_* \sigma^2_z}{k_b}}
\end{equation}


\subsection*{Problem 1C}
Now use the Poisson equation to solve for $\Phi(z)$.

\begin{equation}
    \nabla^2 \Phi = 4\pi G \rho
\end{equation}

With our given by $\rho = m_* n(\vec{R})$, since the system is uniform in the $xy$ plane

\begin{equation}
    \frac{d^2\Phi}{dz^2} = 4\pi Gm_*n(z)
\end{equation}

Let's attack this!

\begin{equation}
    \frac{d^2\Phi}{dz^2} = 4\pi Gm_*n_0 e^{-\frac{\Phi(z)}{\sigma_z^2}}
\end{equation}

Let's redefine the part in the exponent to be:
\begin{equation}
    \aleph = \frac{\Phi}{\sigma_z^2}
\end{equation}

\begin{equation}
    \frac{d^2\aleph}{dz^2} = \frac{4\pi Gm_*n_0}{\sigma_z^2} e^{- \aleph}
\end{equation}

We recognize the scale height as $\boxed{h^2 = \frac{\sigma_z^2}{2\pi Gm_*n_0}}$

\begin{equation}
    \frac{d^2\aleph}{dz^2} = \frac{2}{h^2} e^{- \aleph}
\end{equation}

\begin{equation}
    \frac{d^2\aleph}{dz^2}\frac{d\aleph}{dz} = \frac{2}{h^2} e^{- \aleph}\frac{d\aleph}{dz}
\end{equation}

Integrate:
\begin{equation} \label{eq:1c_halfway}
    \frac{1}{2}(\frac{d\aleph}{dz})^2 = -\frac{2}{h^2} e^{- \aleph} +c
\end{equation}

at $z = 0$, we expect $\frac{d\aleph}{dz} = 0 $ due to symmetry, and we have $\aleph(0) = 0$.

Therefore:
\begin{equation}
   0 = -\frac{2}{h^2} +c \Rightarrow c =  \frac{2}{h^2}
\end{equation}

Throwing back into equation \ref{eq:1c_halfway} we have:
\begin{equation}
    \frac{d\aleph}{dz} = \frac{2}{h} \sqrt{1 - e^{-\aleph}}
\end{equation}

The anti-derivative of this is $\mathrm{sech}$ (from a table).

\begin{equation}
    \boxed{n(z) = n_0 \mathrm{sech}^2(\frac{z}{2h})}
\end{equation}

with $\boxed{h^2 = \frac{\sigma_z^2}{2\pi Gm_*n_0}}$







